\documentclass{beamer}
\usetheme{Dresden}
\author[nemo,mastergreg]{Greg (mastergreg) ~Liras and John (nemo)
~Giannelos}
\institute{foss.ntua}
\title{Python Tutorial}
\subtitle{Part I}

\date{\today}

\begin{document}
\frame{\titlepage}
\section[Intro]{}
\frame{\tableofcontents}
\section{Introduction to Python}
\subsection{Introduction}
\begin{frame}
	\frametitle{Python Tutorial}

Python in a few words
\begin{itemize}
\item<1-> It is is \emph{Scripting Language}
\item<2-> It is \emph{Strongly Typed}
\item<3-> It is \emph{Dynamic}
\item<4-> It is \emph{Portable}
\item<5-> It is \emph{Object Oriented}
\item<6-> It has \emph{Vast Libraries}
\item<7-> It is \emph{Simple and non-obtrucive}
\end{itemize}

\end{frame}

\subsection{Why Python?}
\begin{frame}
	\frametitle{Why?}
	\begin{itemize}
	\item<1-> It is easy to remember
	\item<2-> You can develop rapidly
	\item<2-> Interface with C libraries
	\end{itemize}
\end{frame}
\subsection{Dos and Don'ts}
\begin{frame}
	\frametitle{Must and Must Not}

\begin{itemize}
\item<1-> Must search first
\item<2-> Must not code longer than needed
\item<3-> Must mport only what you need
\item<4-> Must run pychecker on your code
\end{itemize}

\end{frame}

\end{document}
