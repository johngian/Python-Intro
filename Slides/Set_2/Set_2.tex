\documentclass{beamer}
\usepackage{minted}
\usetheme{Dresden}
\author[nemo,mastergreg]{Greg (mastergreg) Liras, John (nemo) Giannelos}
\institute{foss.ntua}
\title{Django Tutorial}
\subtitle{Part II}

\date{\today}


\begin{document}
\AtBeginSection[]
{
\begin{frame}
\frametitle{Outline}
\tableofcontents[currentsection]
\end{frame}
}
%--- the titlepage frame -------------------------%
\begin{frame}
\titlepage
\end{frame}


\section{Introduction to Django Web-Framework}

\subsection{What is Django?}
\begin{frame}
  \frametitle{What is Django?}
\begin{itemize}[<+->]
  \item Django is a web-framework written in Python 
  \item It encourages rapid development and clean pragmatic design 
  \item Follows the MVC (Model-View-Controller) architecture
  \item Focuses on automating common actions (DRY principle) 
  \item Allows the use of pluggable applications
\end{itemize}
\end{frame}

\begin{frame}
  \frametitle{But first of all...}
What is a web-framework?
\end{frame}

\subsection{About web-frameworks.}

\begin{frame}
  \frametitle{What is a web-framework?}
As quoted from wikipedia:
\begin{quote}
A web application framework is a software framework that is designed to support the development of dynamic websites, web applications and web services. The framework aims to alleviate the overhead associated with common activities performed in Web development.
\end{quote}
\end{frame}

\begin{frame}
\begin{itemize}[<+->]
  \item At first, WWW was only a bunch of hardcoded \textsc{html} pages.
  \item Need for more and more dynamic content and user interaction.
  \item New languages designed specifically for web use.
  \item Common tasks-practices showed up in web development.
\end{itemize}
\end{frame}

\begin{frame}
\begin{itemize}
  \item At first, WWW was only a bunch of hardcoded \textsc{html} pages.
  \item Need for more and more dynamic content and user interaction.
  \item New languages designed specifically for web use.
  \item Common tasks-practices showed up in web development.
\end{itemize}
As a result, complete software-stacks appeared to support the web development process. Think of it is a collection of libraries that do common tasks in web development and are designed to work together.
\end{frame}

\begin{frame}
  \frametitle{The MVC architecture}
\begin{itemize}[<+->]
  \item MVC stands for Model-View-Controller
  \item Seperates the modeling of the domain, the presentation and the logic/actions of the project, permitting independent development of each.
    \begin{itemize}
      \item Model: Database and data logic/behaviour
      \item View: (What it says), manages what user views
      \item Controller: Interpretation of user's actions
    \end{itemize}
  \item Django implements MVC slightly different
\end{itemize}
\end{frame}

\subsection{Django basics.}
\begin{frame}
  \frametitle{Django's MVC}
As mentioned before, in the Django framework things are a little bit different. 
\begin{center}
\textbf{Welcome to ``MTV'' (model-template-view).}
\end{center}
\end{frame}

\begin{frame}
  \frametitle{Django's MVC}
\begin{itemize}[<+->]
  \item Model: Pretty much the same as MVC. Describes our database using ORM.
  \item Template: Seperates our website content from its presentation ``style''.
  \item View: Describes which data should be presented (decided using appropriate functions) and sends them to our template. It also takes care of the url handling.
    \begin{itemize}
      \item According to Django's FAQ page: \begin{quote}...a “view” is the Python callback function for a particular URL, because that callback function describes which data is presented. \end{quote}
    \end{itemize}
\end{itemize}
\end{frame}

\begin{frame}
  \frametitle{Django admin-site}
Django has a built-in website for various (common) administrative jobs. It allows you:
\begin{itemize}
  \item To manage site users
  \item To change form display and handling
  \item To make various validations (add,remove,publish e.g ``articles'')
\end{itemize}
\end{frame}

\end{document}
